\documentclass[]{article}
\usepackage{lmodern}
\usepackage{amssymb,amsmath}
\usepackage{ifxetex,ifluatex}
\usepackage{fixltx2e} % provides \textsubscript
\ifnum 0\ifxetex 1\fi\ifluatex 1\fi=0 % if pdftex
  \usepackage[T1]{fontenc}
  \usepackage[utf8]{inputenc}
\else % if luatex or xelatex
  \ifxetex
    \usepackage{mathspec}
    \usepackage{xltxtra,xunicode}
  \else
    \usepackage{fontspec}
  \fi
  \defaultfontfeatures{Mapping=tex-text,Scale=MatchLowercase}
  \newcommand{\euro}{€}
\fi
% use upquote if available, for straight quotes in verbatim environments
\IfFileExists{upquote.sty}{\usepackage{upquote}}{}
% use microtype if available
\IfFileExists{microtype.sty}{%
\usepackage{microtype}
\UseMicrotypeSet[protrusion]{basicmath} % disable protrusion for tt fonts
}{}
\usepackage{color}
\usepackage{fancyvrb}
\newcommand{\VerbBar}{|}
\newcommand{\VERB}{\Verb[commandchars=\\\{\}]}
\DefineVerbatimEnvironment{Highlighting}{Verbatim}{commandchars=\\\{\}}
% Add ',fontsize=\small' for more characters per line
\newenvironment{Shaded}{}{}
\newcommand{\KeywordTok}[1]{\textcolor[rgb]{0.00,0.44,0.13}{\textbf{{#1}}}}
\newcommand{\DataTypeTok}[1]{\textcolor[rgb]{0.56,0.13,0.00}{{#1}}}
\newcommand{\DecValTok}[1]{\textcolor[rgb]{0.25,0.63,0.44}{{#1}}}
\newcommand{\BaseNTok}[1]{\textcolor[rgb]{0.25,0.63,0.44}{{#1}}}
\newcommand{\FloatTok}[1]{\textcolor[rgb]{0.25,0.63,0.44}{{#1}}}
\newcommand{\CharTok}[1]{\textcolor[rgb]{0.25,0.44,0.63}{{#1}}}
\newcommand{\StringTok}[1]{\textcolor[rgb]{0.25,0.44,0.63}{{#1}}}
\newcommand{\CommentTok}[1]{\textcolor[rgb]{0.38,0.63,0.69}{\textit{{#1}}}}
\newcommand{\OtherTok}[1]{\textcolor[rgb]{0.00,0.44,0.13}{{#1}}}
\newcommand{\AlertTok}[1]{\textcolor[rgb]{1.00,0.00,0.00}{\textbf{{#1}}}}
\newcommand{\FunctionTok}[1]{\textcolor[rgb]{0.02,0.16,0.49}{{#1}}}
\newcommand{\RegionMarkerTok}[1]{{#1}}
\newcommand{\ErrorTok}[1]{\textcolor[rgb]{1.00,0.00,0.00}{\textbf{{#1}}}}
\newcommand{\NormalTok}[1]{{#1}}
\ifxetex
  \usepackage[setpagesize=false, % page size defined by xetex
              unicode=false, % unicode breaks when used with xetex
              xetex]{hyperref}
\else
  \usepackage[unicode=true]{hyperref}
\fi
\hypersetup{breaklinks=true,
            bookmarks=true,
            pdfauthor={},
            pdftitle={},
            colorlinks=true,
            citecolor=blue,
            urlcolor=blue,
            linkcolor=magenta,
            pdfborder={0 0 0}}
\urlstyle{same}  % don't use monospace font for urls
\setlength{\parindent}{0pt}
\setlength{\parskip}{6pt plus 2pt minus 1pt}
\setlength{\emergencystretch}{3em}  % prevent overfull lines
\setcounter{secnumdepth}{0}

\date{}

\begin{document}

\section{Programmation orientée
Objet}\label{programmation-orientuxe9e-objet}

\begin{quote}
Notes de cours
\end{quote}

{[}\ldots{}{]}

\subsection{Concepts Objets}\label{concepts-objets}

{[}\ldots{}{]}

\subsubsection{Extension}\label{extension}

\paragraph{Héritage}\label{huxe9ritage}

\emph{SuperClasse} -\textgreater{} SousClasse (Hérite des attributs)

\paragraph{Redéfinition et
surcharge}\label{reduxe9finition-et-surcharge}

\emph{Redéfinition} : Classe hérite d'une méthode mais on change son
action (\texttt{@Override}).

\emph{Surcharge} : Plusieurs fois la même méthode avec paramètres
différents.

\paragraph{Classe abstraite et
interface}\label{classe-abstraite-et-interface}

Une \textbf{classe abstraite} , n'est pas instaciable, elle sert
uniquement à définir un super-type qui pourra être refédini.

Mot-Clef : \texttt{abstract}

Une \textbf{méthode abstraite} est aussi sans implémentation.

\begin{Shaded}
\begin{Highlighting}[]
\KeywordTok{abstract} \KeywordTok{class} \NormalTok{ClassAbstraite \{}
    \KeywordTok{public} \DataTypeTok{int} \FunctionTok{methode}\NormalTok{() \{}
        \KeywordTok{return} \DecValTok{42}\NormalTok{;}
    \NormalTok{\}}
    \KeywordTok{public} \KeywordTok{abstract} \NormalTok{String }\FunctionTok{methodeAbstraite}\NormalTok{(}\DataTypeTok{int} \NormalTok{a);}
\NormalTok{\}}
\end{Highlighting}
\end{Shaded}

Une \textbf{interface} est un ensemble de signatures de méthodes, elle
n'a pas d'implémentation. Elle définie un \emph{type}. Une classe peut
implémenter plusieurs interfaces, elle fournit une
\textbf{implémentation} aux méthodes de l'interface.

Une interface est équivalent à une classe ``purement'' abstraite.
Mots-clés : \texttt{interface,\ implements}

Convention de nommage : - Bof : I\emph{ - Bien : }able

\begin{Shaded}
\begin{Highlighting}[]
\KeywordTok{interface} \NormalTok{Fooable \{}
    \NormalTok{String }\FunctionTok{foo}\NormalTok{(}\DataTypeTok{int} \NormalTok{a);}
\NormalTok{\}}

\KeywordTok{class} \NormalTok{MyFoo }\KeywordTok{implements} \NormalTok{Fooable \{}
    \KeywordTok{public} \NormalTok{String }\FunctionTok{foo}\NormalTok{(}\DataTypeTok{final} \DataTypeTok{int} \NormalTok{a) \{}
        \KeywordTok{return} \StringTok{"foo : "} \NormalTok{+ a;}
    \NormalTok{\}}
\NormalTok{\}}
\CommentTok{/*...*/}
\end{Highlighting}
\end{Shaded}

Séparation interface / implémentation, dépendance sur interface
=\textgreater{} Découplage Interfaces minimales -\textgreater{}
sous-typages structurelS

\begin{Shaded}
\begin{Highlighting}[]
\KeywordTok{interface} \NormalTok{Iterable\{}\CommentTok{/* */}\NormalTok{\}}

\KeywordTok{class} \NormalTok{MyList }\KeywordTok{implements} \NormalTok{Iterable \{}\CommentTok{/* */}\NormalTok{\}}

\KeywordTok{class} \NormalTok{Client \{}
    \KeywordTok{private} \NormalTok{Iterable elements;}
    \CommentTok{/* */}
\NormalTok{\}}
\end{Highlighting}
\end{Shaded}

\paragraph{Délégation}\label{duxe9luxe9gation}

\begin{itemize}
\itemsep1pt\parskip0pt\parsep0pt
\item
  Réutilisation
\item
  Sans sous-typage
\item
  Sorte d'alternative à l'héritage
\end{itemize}

Principe : Préférer la composition à l'héritage

\end{document}
